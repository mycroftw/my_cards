%LaTeX convention card editor

%v1.0 - Gordon Bower - 17 December 2017
%v1.0.1 - alignment fixes, default text formatting in style file
%v2.0 - Michael Farebrother - convert to 2022 card - Feb 24 2022

%Fill out the fields below to create your own ACBL Convention Card in LaTeX
%acbl2022cc.sty contains the TikZ formatting instructions

%grbcce requires the following packages: tikz, txfonts, microtype, ifthen, xstring

\documentclass[12pt]{article}
%the card is 8'' wide, so the margins are very narrow to fit on letter size paper:
\usepackage{acbl2022cc}
\usepackage[left=0mm,right=5mm,top=20mm]{geometry}
\usepackage{enumitem}
\begin{document}

%There is a boolean to turn on or off every checkbox on the card.
%To fill in any checkbox, remove the ``%'' in front of the \setboolean statement
%that activates that particular checkbox.
%You may delete unused setboolean statements.

%There is a text item to insert a description on every blank line of the card.
%You WILL get errors if you delete any \newcommand statements!
%if you want to leave a line blank, use \newcommand{\nameoftextfield}{} .

%If you insert plain text into any text field, default formatting is applied.
%To override my default, use any of the following predefined styles:
%\regtext{...} for regular black text
%\redreg{...} for regular red text (for conventions you will alert)
%\bluereg{...} for regular blue text (for announcements)
%\boldtext{...} for bold black text
%\redbold{...} for bold red text
%\bluebold{...} for bold blue text (e.g. more visible 1NT ranges)
%Use whatever other LaTeX formatting you wish. Default font size is \scriptsize.

% To insert suit symbols: \s, \h, \d, \c.
% To insert a naturally coloured suit: \bc, \rd, \rh, \bs
%To insert a red spade: \textcolor{red}{\s}

%To specify what card you lead from a given holding, indicate that card's
%POSITION in the given sequence:
%e.g. if you lead T from KJTx, \newcommand{\KTJx}{3}
%You MAY circle more than one card if you wish. e.g., \newcommand{\TNxx}{14}
%If you use the default, you may choose to circle none, \newcommand{\TNxx}{}

%You WILL get errors if you delete any \newcommand statements!
%If you lead the default card, you may use \newcommand{cardcombination}{}.

% The following two booleans determine the formatting of the card.
%     If visib is true, the guidelines under the user text are printed.
%     Comment this out to remove them (looks cleaner, but may seem odd)
%
%     If serif is true, then user text is in a serif font, to differentiate
%     if from pre-printed card text (this is especially useful when
%     guidelines are off).  If false, user text is the same sans-serif font
%     as the pre-printed text.  Pick exactly one of the three lines for true,
%     false, or "serif if no guidelines, sans if guidelines" (the default).
%
\setboolean{visib}{false}  % show guidelines

\ifthenelse{\boolean{visib}}{}{\setboolean{serif}{true}}
%\setboolean{serif}{true}
%\setboolean{serif}{false}

% it is sometimes very hard to read the small text on printed page.
% Uncomment one of these commands if you want to override the default
% (bigger if serif, normal size if not)
%\renewcommand{\usertextsize}{\footnotesize}  % larger answer size
%\renewcommand{\usertextsize}{\scriptsize}    % smaller answer size

% And now the card itself. Right side first, top to bottom, then left side.
\newcommand{\names}{Michael Farebrother -- Ellen Kuiper}
\newcommand{\playernumber}{Q550794/O007899}

%General approach:
\newcommand{\generalapproach}{Weak NT, Keri/NT}
\newcommand{\minHCPopen}{12 (11 M)}
\newcommand{\minHCPresp}{A or 6 (5/m)}
%\setboolean{fo1c}{true} %Forcing openings...
\setboolean{fo2c}{true}
\newcommand{\foothertext}{4\bc, 4\rd}%Other forcing opening bids
%\setboolean{1nts}{true} % 1NT strength
\setboolean{1ntw}{true}
%\setboolean{1ntv}{true}
\newcommand{\bidspreptop}{1NT--2\c\ Keri,\hspace{2mm}Upside Down Suit Preference} % Bids that Require Preparation
\newcommand{\bidsprepbottom}{1NT O/call takeout, 1 suit-X 15+ ``any''.\hspace{2mm}Notes [\#] on back for detail} % Bids that Require Prep line 2

% 1 Club
%\setboolean{club5}{true} %1C length
%\setboolean{club4}{true}
\setboolean{club3}{true}
%\setboolean{cl2nf}{true}
%\setboolean{c4432}{true}
%\setboolean{cl1nf}{true}
%\setboolean{cl0nf}{true}
%\setboolean{clcon}{true}
\newcommand{\cldescribe}{Sound unless 6+\ \c} % Insert text in blank space next to 1C
\newcommand{\clresponses}{}
%\setboolean{ctxfr}{true} % transfer responses to 1C

\newcommand{\clrespdi}{} % text after "1D response"
\setboolean{bypas}{true}
\newcommand{\minonenlowtext}{8} %1C-1NT range
\newcommand{\minonenhightext}{10}
\newcommand{\mintwonlowtext}{11}
\newcommand{\mintwonhightext}{12}

%\setboolean{clsrn}{true} % Single raise - NF
\setboolean{clsri}{true} %              - Inv+
%\setboolean{clsrf}{true} %              - GF
\setboolean{cljrw}{true} % jump raise - Weak
%\setboolean{cljrm}{true} %            - Mixed
%\setboolean{cljri}{true} %            - Inv
\setboolean{clocw}{true} % After overcall - NF
%\setboolean{clocm}{true} %                - Mixed
%\setboolean{cloci}{true} %                - Inv

% 1 Diamond
%\setboolean{diam5}{true} %1D length
\setboolean{diam4}{true}
\setboolean{diam3}{true}
%\setboolean{dubal}{true}
%\setboolean{di2nf}{true}
%\setboolean{di1nf}{true}
%\setboolean{di0nf}{true}
%\setboolean{dicon}{true}
\newcommand{\didescribe}{4 unless 4--4--3--2, sound unless 6+\ \rd} % Insert text in blank space next to 1D e.g. ``4 unless 4-4-3-2''
\newcommand{\diresponses}{}
\setboolean{dascl}{true} % 1D responses like 1C

\newcommand{\minonedionenlowtext}{6} %1D-1NT range
\newcommand{\minonedionenhightext}{9}
\newcommand{\minoneditwonlowtext}{10}
\newcommand{\minoneditwonhightext}{12}

%\setboolean{disrn}{true} % Single raise - NF
\setboolean{disri}{true} %              - Inv+
%\setboolean{disrf}{true} %              - GF
\setboolean{dijrw}{true} % jump raise - Weak
%\setboolean{dijrm}{true} %            - Mixed
%\setboolean{dijri}{true} %            - Inv
\setboolean{diocw}{true} % After overcall - NF
%\setboolean{diocm}{true} %                - Mixed
%\setboolean{dioci}{true} %                - Inv

% 1 Major
%\setboolean{h12l4}{true} %4 cards in 1st/2nd seat
\setboolean{h12l5}{true}
\setboolean{h34l4}{true} %4 cards in 3rd/4th seat
\setboolean{h34l5}{true}
\setboolean{ntfor}{true} %1NT forcing
%\setboolean{ntsem}{true} %1NT semiforcing
%\setboolean{ntbps}{true} %1H-1NT bypass 4+S
\newcommand{\majothertexttop}{Kokish GT} %Other major methods: HSGT, Mathe asking bid, etc
\newcommand{\majothertextbottom}{\regtext{2/1 GF unless Suit Rebid}} %Other major methods line two

\setboolean{ra2nt}{true} %Jacoby
\setboolean{ra3nt}{true} %1M-3NT
\setboolean{raspl}{true} %Splinter
\newcommand{\majraisetext}{Fit J/S (Limit-ish)} %Text line for Bergen, fitjumps, etc
\setboolean{dru2c}{true} % 2 clubs Drury?
%\setboolean{dru2d}{true} % 2 diamonds Drury?
%\setboolean{drcmp}{true}
\newcommand{\majdrurytext}{}
%\setboolean{dblrw}{true} %double raise
%\setboolean{dblrm}{true}
\setboolean{dblri}{true}
\setboolean{ocdrw}{true} %double raise in comp
%\setboolean{ocdrm}{true}
%\setboolean{ocdri}{true}

% 1NT: Top
\newcommand{\ntlowtext}{11++}
\newcommand{\nthightext}{14}
\newcommand{\ntseatvul}{}
\newcommand{\altntlowtext}{} %If variable NT, second range here
\newcommand{\altnthightext}{}
%\setboolean{ntasr}{true} %Same responses for alt NT range?
% This next line is "differences between ranges" if a second range defined,
% or a space for "style" if only one range.
\newcommand{\altntresponse}{could have stiff A/K in a minor, could be 6m322~\boldtext{[\ref{itm:1nt}]}}

% 1NT: First column
%\setboolean{nt5cm}{true}
\newcommand{\sysontext}{2\,\c} %Systems on after 1NT-(X), 2\c, etc...
%\setboolean{staym}{true}
%\setboolean{puppe}{true}
\setboolean{2coth}{true}
%\setboolean{nat2d}{true}
\setboolean{jac2d}{true}
\newcommand{\ntditext}{v. rare 4\,\rh~\boldtext{[\ref{itm:1ntother}]}} %Other meaning for 1NT-2\d
%\setboolean{nat2h}{true}
\setboolean{jac2h}{true}
\newcommand{\nthetext}{} %Other meaning for 1NT-2\h
%\setboolean{nat2s}{true}
%\setboolean{trf2s}{true}
\newcommand{\ntsptext}{range ask~\boldtext{[\ref{itm:1ntother}]}} %1NT-2S
%\setboolean{nat2n}{true}
\setboolean{trf2n}{true}
\newcommand{\ntnttext}{\regtext{clubs}} %1NT-2NT

% 1NT: second column
\newcommand{\ntcjumptext}{\rd\ \redreg{splinter~\boldtext{[\ref{itm:1ntother}]}}} %1NT-3C
\newcommand{\ntdjumptext}{\rh\ \redreg{splinter}}
\newcommand{\nthjumptext}{\s\ \redreg{splinter}}
\newcommand{\ntsjumptext}{\c\ \redreg{splinter}} %1NT-3S
\newcommand{\ntothertexttop}{2\,\bc\ forces 2\,\rd, to play or INV~\boldtext{[\ref{itm:1nt2c}]}} %Other conventions over 1NT
\newcommand{\ntothertextbottom}{\textcolor{black}{1NT--X:} 2\,\bc\ scramble, \textcolor{black}{else to play}~\boldtext{[\ref{itm:1ntx}]}} % second "Other Conventions"

% 1NT: Bottom
%\setboolean{smole}{true}
\setboolean{tex4c}{true}
\setboolean{tex4d}{true}
%\setboolean{tex4h}{true}
\setboolean{ntneg}{true}
\newcommand{\ntnegtext}{} %Negative doubles over 1NT?
%\setboolean{ntpen}{true}
\newcommand{\ntxothertext}{} %Other doubles over 1NT
\setboolean{leben}{true}
\newcommand{\ntlebentext}{fast denies} %Lebensohl style

% 2NT
\newcommand{\twonlowtext}{20}
\newcommand{\twonhightext}{21}
%\setboolean{pup2n}{true}
\newcommand{\twonsptext}{one or two-minor slam try} %2NT-3S
%\setboolean{con2n}{true}
\newcommand{\convnttext}{} % Conventional 2NT
\setboolean{jac2n}{true}
\setboolean{tex2n}{true}
\setboolean{neg2n}{true}
\newcommand{\twonothertext}{\bluereg{SA Texas}} %Other line in 2NT box

% 3NT
\newcommand{\threenlowtext}{} %3NT range
\newcommand{\threenhightext}{}
\setboolean{gam3n}{true} % One suit?
\newcommand{\threenonesuittext}{4m preempt, NOT solid} %One suit description 3NT box

% 2 Clubs
\newcommand{\twoclowtext}{22+} %2C point range
\newcommand{\twochightext}{}
\newcommand{\twocldescribe}{or 8.5 QT w/Defence} %line of text on left side of 2C divider
\setboolean{vs2cl}{true} %2C Very strong?
%\setboolean{st2cl}{true}
%\setboolean{na2cl}{true} %2C Natural?
%\setboolean{ot2cl}{true} %2C conventional?
\newcommand{\twoclconvtext}{}
%\setboolean{neg2d}{true} %2D neg or waiting?
\setboolean{wai2d}{true}
%\setboolean{steps}{true} %step/control responses?
\newcommand{\twoclstepstext}{}
%\setboolean{neg2h}{true}
\newcommand{\twoclresponse}{cheapest 3, x, xx 2-neg} %Other text on right side of 2C divider

% 2 Diamonds
\newcommand{\twodlowtext}{6} %2D point range
\newcommand{\twodhightext}{11}
\newcommand{\twoddescribe}{Anything goes style~\boldtext{[\ref{itm:2x}]}}
%\setboolean{dnsnf}{true} %new suit NF
\setboolean{wk2di}{true} %2D weak
%\setboolean{in2di}{true}
%\setboolean{st2di}{true}
%\setboolean{co2di}{true} %2D conventional
\newcommand{\twodntresp}{Ogust}
\newcommand{\twodresponse}{Fit Raise/X...}

% 2 Hearts
\newcommand{\twohlowtext}{6}
\newcommand{\twohhightext}{10}
\newcommand{\twohdescribe}{as 2\,\rd}
%\setboolean{hnsnf}{true}
\setboolean{wk2he}{true}
%\setboolean{in2he}{true}
%\setboolean{st2he}{true}
%\setboolean{2s2he}{true}
\newcommand{\twohntresp}{Ogust}
\newcommand{\twohresponse}{Parking XX}

% 2 Spades
\newcommand{\twoslowtext}{6}
\newcommand{\twoshightext}{10}
\newcommand{\twosdescribe}{as 2\,\rd}
%\setboolean{snsnf}{true}
\setboolean{wk2sp}{true}
%\setboolean{in2sp}{true}
%\setboolean{st2sp}{true}
%\setboolean{2s2sp}{true}
\newcommand{\twosntresp}{Ogust}
\newcommand{\twosresponse}{[\ref{itm:2xdoubled}]}

%Bottom of card
\newcommand{\jshifttext}{Fit (Limit-ish, 9+ in two suits)~\boldtext{[\ref{itm:fjs}]}} % jump shift responses
\newcommand{\vsstrongart}{Mathé/1m} % "vs str/VS openings" - MDF to fix
%\setboolean{nmf1w}{true} %1-way new minor
%\setboolean{nmf2w}{true} %2-way new minor
\setboolean{xyz2w}{true} %XYZ
%\setboolean{fsfrd}{true} %Fourth suit
\setboolean{fsfgf}{true}
\newcommand{\footnotestop}{Wolff Signoff after 1\,x--1\,y; 2NT} % 2 lines for additional conventions at bottom
\newcommand{\footnotesbottom}{Very aggressive competition against 2-fit}

% Back of card (left to right, top to bottom)

% Doubles
\setboolean{negax}{true} %Negative doubles through...
\newcommand{\negaxthru}{3\,\s}
%\setboolean{penax}{true}
\setboolean{respx}{true} %Responsive
\newcommand{\respxthru}{3\,\s}
\setboolean{maxix}{true} %Maximal
\setboolean{suppx}{true} %Support
\newcommand{\suppxthru}{2x (-1NT)}
\setboolean{supxx}{true} %Support redouble
\newcommand{\toxstyletext}{\redreg{15+``any''(xx+ if min.)~\boldtext{[\ref{itm:pd}]}}}
\newcommand{\doubleothertext}{see 1NT for takeout ``double''} %Other special doubles

% NT overcalls
\newcommand{\directlowtext}{\redreg{(6)\,8}} %range
\newcommand{\directhightext}{\redreg{14(18)}}
%\setboolean{syson}{true} %systems on
\newcommand{\ballowtext}{11} %balancing range
\newcommand{\balhightext}{14}
\setboolean{sysba}{true}
\setboolean{dntco}{true} %conventional NT O/c
\newcommand{\directconvtext}{1NT t/o, 3+ in unbids~\boldtext{[\ref{itm:1ntoc}]}}
\setboolean{unu2l}{true} %Unusual: 2 lowers
\newcommand{\ntothertext}{} %conventional overcalls (sandwich etc)

% Overcalls
\newcommand{\oclowtext}{8} %point range
\newcommand{\ochightext}{14(18)}
%\setboolean{oc4cd}{true}
\newcommand{\twooclowtext}{11} %2-level point range
\newcommand{\twoochightext}{14(18)}
%\setboolean{jostr}{true} %Jump overcalls strong
%\setboolean{joint}{true}
\setboolean{jowea}{true} %weak
%\setboolean{jocon}{true}
\newcommand{\jumpovercalltext}{0-1 outside AK controls} %Other info about jump overcalls

\setboolean{nsfor}{true} %New suit forcing
%\setboolean{nsnfc}{true} %NF Constructive
%\setboolean{nsnon}{true} %NF
%\setboolean{onstf}{true} %transfer
\setboolean{jrwea}{true}
%\setboolean{jrmix}{true}
%\setboolean{jrinv}{true}
\newcommand{\overcallcuetext}{Limit+}
\setboolean{occus}{true} % cue response support?
\newcommand{\overcallothertext}{} % Other responses

%Defense vs. NT
\newcommand{\firstcondition}{Strong/All} %''vs.'' line, first half
\newcommand{\firstvsdbl}{\redreg{usu. 4M--5+m}} %1NT-X
\newcommand{\firstvstwocl}{\rh\,\redreg{ + }\bs} %1NT-2C
\newcommand{\firstvstwodi}{\rh\,\redbold{ or }\bs} %1NT-2D
\newcommand{\firstvstwohe}{\redreg{5\,\rh\, +\, m}} %1NT-2H
\newcommand{\firstvstwosp}{\redreg{5\,\bs\, +\, m}} %1NT-2S
\newcommand{\firstvstwont}{\redreg{\bc\,+\,\rd}} %1NT-2NT
\newcommand{\secondcondition}{Weak (< 16)} %the same six lines, right half
\newcommand{\secondvsdbl}{Penalty}
\newcommand{\secondvstwocl}{}
\newcommand{\secondvstwodi}{}
\newcommand{\secondvstwohe}{\redreg{\rh\, + m}}
\newcommand{\secondvstwosp}{\redreg{\bs\, + m}}
\newcommand{\secondvstwont}{}
\newcommand{\othernt}{} %two lines of other NT defense info

%Direct cuebid
%\setboolean{qamim}{true} %Michaels - Art minor
\setboolean{qqmim}{true} %         - Quasi Natural minor
\setboolean{qnmim}{true} %         - Natural minor
\setboolean{qnmam}{true} %         - Natural Major
\setboolean{qamin}{true} %Natural
%\setboolean{qqmin}{true}
%\setboolean{qnmin}{true}
%\setboolean{qnman}{true}
%\setboolean{qamio}{true} %Other
%\setboolean{qqmio}{true}
%\setboolean{qnmio}{true}
%\setboolean{qnmao}{true}
\newcommand{\cuebiddescribe}{Mathé/Str, std/multi-1m} % other cuebid comments

% vs. takeout double
%\setboolean{nsf2l}{true} % New suit forcing at 2 level?
%\setboolean{xnstf}{true} %New suit transfer?
\newcommand{\toxtrftext}{} %transfers through?
%\setboolean{jswea}{true} %Jump shifts
%\setboolean{jsinv}{true}
%\setboolean{jsfor}{true}
\setboolean{jsfit}{true}
\setboolean{xximp}{true} %Rdbl 10+
%\setboolean{xxcon}{true}
\newcommand{\xxtext}{} % Conventional XX explanation
%\setboolean{minnt}{true} % 1m-X-2NT Nat
\setboolean{minrs}{true}
\newcommand{\xtwontminormin}{6}
\newcommand{\xtwontminormax}{9}
%\setboolean{majnt}{true}
\setboolean{majrs}{true}
\newcommand{\xtwontmajormin}{10}
\newcommand{\xtwontmajormax}{+}
\newcommand{\toxothertext}{} %Line at bottom of Opps TOX section

% Opening preempts
\newcommand{\threelevelstyletop}{Agg. (esp NV)}
\newcommand{\threelevelstylebottom}{3m 1\textsuperscript{st}/2\textsuperscript{nd} ``happy to hear 3NT''~\boldtext{[\ref{itm:3m}]}}
\newcommand{\threelevelresp}{NSF}
\newcommand{\fourlevelstyle}{4m 0- or 1-loser M+A/K}
\newcommand{\fourlevelresp}{gap asks which}
\setboolean{namya}{true}
\newcommand{\fourlevelother}{}

%vs. opening preempts
\newcommand{\optwontoc}{15--17 Balanced} % (2x)-2NT
\setboolean{opdto}{true}
\newcommand{\opdtothru}{4\,\rh} %Takeout through...
%\setboolean{oppen}{true}
\setboolean{opleb}{true} %Lebensohl
\newcommand{\oplebtext}{} %leb text
\newcommand{\opcuebid}{}
\newcommand{\opjumpoc}{}
\newcommand{\opothertext}{} %Other line

%Slam conventions
%\setboolean{gednt}{true} %Gerber
%\setboolean{geseq}{true} %NT Seq G
%\setboolean{genon}{true} %non-seq G
\newcommand{\gerbertext}{\textbf{Never}}
%\setboolean{black}{true} %Blackwood
\setboolean{b0314}{true} %RKC
%\setboolean{b1430}{true} %1430
\newcommand{\blackwoodtext}{0123 first round}
\newcommand{\controlcuetext}{1\textsuperscript{st}/2\textsuperscript{nd} up the line, Frivolous 3NT/M}
\newcommand{\slaminterftext}{DFS/RFS/PSS, DEPO $\rightarrow$ 5 trump}
\newcommand{\slamothertext}{1NT--2\,\s; resp-3\,x RKC} %Two lines for slam convention info

% Carding
%\setboolean{stasu}{true} %std attitude vs suits
%\setboolean{stcsu}{true} %count
\setboolean{udasu}{true} %upside down attitude suits
\setboolean{udcsu}{true} %count
%\setboolean{stant}{true} %std attitude vs NT
%\setboolean{stcnt}{true} %count
\setboolean{udant}{true} %upside down attitude NT
\setboolean{udcnt}{true} %count
\newcommand{\cardexcepttext}{}
%\setboolean{smisu}{true} %Smith echo - suit
\setboolean{smint}{true} %Smith - NT
\setboolean{smirv}{true} %reverse Smith
\newcommand{\othercardingtext}{\boldtext{Upside Down Suit Pref.~\boldtext{[\ref{itm:carding}]}}}
\newcommand{\trumpsignaltext}{(UD)SP}

% Signals
% Two alternatives here:
%    1. uncomment the setboolean to check the box that is "primary signal", as the card suggests.
%    2. Do not uncomment any checkbox, but rank the entries by putting a single
%       character in each box with the {*rank} commands.
%\setboolean{deatt}{true} %Primary signal to declarer's lead - attitude
\setboolean{decou}{true} % count
%\setboolean{desui}{true} % suit preference
\newcommand{\deattrank}{}
\newcommand{\decourank}{}
\newcommand{\desuirank}{}
\setboolean{pratt}{true} %Primary signal to partner's lead - attitude
%\setboolean{prcou}{true} % count
%\setboolean{prsui}{true} % suit preference
\newcommand{\prattrank}{}
\newcommand{\prcourank}{}
\newcommand{\prsuirank}{}
\newcommand{\signalexcepttext}{Frequent Suit Pref. Leads}
\newcommand{\signaltexttop}{}
\newcommand{\signaltextbottom}{}

%\setboolean{std1d}{true} % first discard: standard attitude
\setboolean{udn1d}{true} % UD attitude
%\setboolean{lav1d}{true} % Lavinthal
%\setboolean{odd1d}{true} % Odd/Even
%\setboolean{oth1d}{true} % Other

%Leads:
% indicate whether to circle the 1st, 2nd, etc. card (more than one, or zero, is OK)
%e.g. if you lead Ace from Ace-King, \newcommand{AKx}{1}
%e.g. if you lead the default from Queen-Jack, \newcommand{QJx}{}

% Leads vs suit
\setboolean{su4th}{true} %4th best vs. suit
%\setboolean{su3rd}{true} %3/5 suit
%\setboolean{su3lo}{true} %3/low suit
%\setboolean{suatt}{true} %attitude suit
%\setboolean{susxx}{true} %low from xx
\newcommand{\suitxx}{1}
\newcommand{\suitxxx}{}
\newcommand{\suitxxxx}{4}
\newcommand{\suitxxxxx}{4}
\newcommand{\suitHxx}{3}
\newcommand{\suitHxxx}{4}
\newcommand{\suitHxxxx}{4}
\newcommand{\suitafterfirst}{}
\newcommand{\AKx}{12}
\setboolean{suakv}{true} % AK leads vary/suits?
\newcommand{\suitAKxvaries}{A Att., K Kount}
\newcommand{\KQx}{}
\newcommand{\QJx}{}
\newcommand{\JTx}{}
\newcommand{\TNx}{}
\newcommand{\KJTx}{}
\newcommand{\KTNx}{}
\newcommand{\suitQTNx}{}
\newcommand{\suitexcepttop}{K from AK}
\newcommand{\suitexceptbottom}{Pd's suit low: Qxx/support, xxx/not}

% Leads vs. NT
\setboolean{nt4th}{true} %4th best vs. NT
%\setboolean{nt3rd}{true} %3/5 NT
%\setboolean{nt3lo}{true}
%\setboolean{ntatt}{true} %Attitude vs NT
%\setboolean{nt2xx}{true} %2nd from xxxx+
\newcommand{\ntxx}{1}
\newcommand{\ntxxx}{1}
\newcommand{\ntxxxx}{4}
\newcommand{\ntxxxxx}{4}
\newcommand{\ntHxx}{3}
\newcommand{\ntHxxx}{4}
\newcommand{\ntHxxxx}{4}
\newcommand{\ntafterfirst}{}
\newcommand{\AKxx}{12}
\setboolean{ntakv}{true}
\newcommand{\ntAKxvaries}{K Kount/unbl.}
\newcommand{\KQJx}{}
\newcommand{\ntKQTN}{}
\newcommand{\QJTx}{}
\newcommand{\JTNx}{}
\newcommand{\AQJx}{}
\newcommand{\AJTx}{}
\newcommand{\ntKTNx}{}
\newcommand{\ntQTNx}{}
\newcommand{\ntexcepttop}{}
\newcommand{\ntexceptbottom}{}

% END of pre-set blanks on the convention card

% Finally, if you want to insert a completely new field in the place of your
% choice, you may do so by placing additional TikZ commands into this wrapper
% if you don't want to edit the style file.

% If not, leave it blank as it is below.

% To create a new text field:
% \node at (x,y) {Text to position at this location};
% The semicolon is essential!

\newcommand{\additionaltikz}{%
%
}%end of \additionaltikz
\drawconventioncard

\twocolumn[\centerline{\large\textbf{Notes for Michael Farebrother --- Ellen Kuiper: K/S system: \today{}}}]

\begin{scriptsize}
    \begin{enumerate}
        \item \label{itm:1nt}1 NT is 12--14 all seats, all vulnerabilities.
         We upgrade special 11s (Michael more than Ellen).
        \begin{enumerate}[label*=\arabic*,nosep]
            \item \label{itm:1nt2c}We do \textbf{not} play Stayman over 1 NT.  2\,\bc\ is Keri: a puppet to 2\,\rd.
            This is forced; there is no other systemic response to 2\,\bc.
            Responder's rebids:
            \begin{description}
                \item [Pass] Weak with Diamonds
                \item [2\,\rh] INV with 4 \textbf{or 5} hearts (but see 1 NT--2\,\rd below)
                \item [2\,\bs] INV with 4 \textbf{or 5} spades
                \item [2 NT] GF Puppet Stayman.
                 Specialized responses.
                \item [3 m] INV with 6+ of the minor
                \item [3 M] GF, 5+ \textbf{diamonds} and 5 of the major
                \item [3 NT] Mild \textbf{diamond} slam try.
                 Usually 5332 or 6322.
            \end{description}
            \item We do \textbf{not} play Gerber.
            Over 1 NT and 2 NT, 4 m transfers to corresponding major, 4 M is to play.
            \item \label{itm:1ntother}Other special responses over 1NT:
            \begin{description}
                \item [1 NT--2\,\rd] hearts, but very rarely 4: if responder rebids 2\,\bs,
                      they are INV with both majors: 5\,\bs--5\,\rh, 5\,\bs--4\,\rh\ or 4\,\bs--4\,\rh, but not 4\,\bs--5\,\rh.
                \item [1 NT--2\,\bs] Range ask.
                2 NT is minimum, 3\,\bc\ is maximum.
                Responder's suit rebid is RKC (3NT: xx, then 4\,\bc: ``don't care, Keycard.'')
                \item [1 NT--3 suit] GF, 3-suited, short in \textbf{the next suit up}.
                 Bidding the short suit shows no wasted values and max, 3NT promises good stoppers.
                \item [1 NT--4 NT] 0123 Blackwood.
                 \textbf{Not} quantitative.
            \end{description}
            \item \label{itm:1ntx}Over interference:
            \begin{itemize}
                \item Over double (all seats, all meanings), our runout is ``2\,\bc\ Scramble'':
                \begin{itemize}
                    \item Keri and Transfers are off.
                    Almost all calls, including Pass and XX, are natural and to play.
                    XX sets a force through 2\,\rd.
                    \item The exception is 2\,\bc, which is ``Natural'', but 1+.
                     This call is to play undoubled.
                      If doubled:
                    \begin{itemize}
                        \item Opener will pass with 3+ clubs, or bid her lowest 4-card suit.
                        \item Responder will pass with 4+ clubs, or bid her lowest 4-card suit.
                        \item Runs are also to play unless doubled, or if a 7 card fit is found.
                        Further bidding as above.
                    \end{itemize}
                \end{itemize}
                \item Over 2\,\bc\ overcalls (again, no matter what they mean), systems on, X is Keri.
                \item Over 2\,\rd\ or higher, Lebensohl, double of shown suit is Negative.
                \item When dealing with a forcing ``X or Y'' interference, we may choose to bid directly or allow you to clarify and Lebensohl over that.
            \end{itemize}
        \end{enumerate}
        \item \label{itm:fjs}We play Fit Jump Shifts (9+ cards in bid suit and partner's, about limitish values) even by unpassed hand without competition.
        \item \label{itm:michaels}In situations where we show two known or one known and one unknown suit, 2 NT is strong and asking, and cheapest unknown suit is pass/correct.
            This applies to:
            \begin{itemize}[nosep]
                \item Woolsey overcalls of 1 NT
                \item Michaels cuebids
            \end{itemize}
        \newpage
        \item We play Power Doubles and 1NT Overcall for takeout:
        \begin{enumerate}[label*=\arabic*,nosep]
            \item \label{itm:pd}(1 suit)--X shows ``any'' 15+, but because we will pass for penalty with much weaker trumps than ``normal'', with a singleton, it's about 17, and with a void, about 19.
            \begin{itemize}
                \item It frequently shows a strong NT, with or without stoppers.
                \item Next suit up (exc.\ 1\,\bs--X) is an artificial negative, 0--bad 4 any.
                \item Non-jump bids are natural and about good 4--bad 8.
                \item 1 NT is good 4--7 basically balanced, and implies, but does not promise, a stopper.
                \item Cuebid is good 8+, GF
                \item Jump bids are good suits.
            \end{itemize}
            \item \label{itm:1ntoc}(1 suit)--1 NT is takeout, usually 8--14 HCP, guaranteeing 3+cards in all unbid suits.
            \begin{itemize}
                \item With perfect shape, especially NV in 2nd, could be on 6 (Open chart only).
                \item With a singleton (16) or void (18) in bid suit, we may overcall 1NT with more than 14 HCP\@.
                \item We are fairly aggressive with this call, and can do it with 5 card suits; but with 4333, we will be on the top of the range.
                \item Pass is rare, but possible.
                \item New suits are natural and to play.
                \item Cuebid is Stayman-esque, and could be the start of a INV+ sequence.
                \item Jump bids are preemptive [Exception: INV at unfavourable].
                \item If 1 NT is doubled, new suits are to play (usually 5+), XX shows 4 cards in the highest unbid suit, and pass is any other (\textbf{not} to play).
            \end{itemize}
        \end{enumerate}
        \item \label{itm:preempt}Preempts:
        \begin{enumerate}[label*=\arabic*,nosep]
            \item Our preempt style is more aggressive than most, heavily dependent on vulnerability and seat:
            \begin{itemize}[nosep]
                \item \label{itm:2x}NV first seat weak twos could be Jxxxxx.
                      Second seat vulnerable, the minimum is probably KJ9xxx.
                \item \label{itm:3m}Similar arguments apply to 3- and 4-major bids, and 3NT (4 level minor preempt, \textbf{not} gambling).
                \item However, 3\,\bc/\rd\ in first and second seat are ``happy to put dummy down in 3NT'', and 4\,\bc/\rd\ is Namyats (solid 8 and nothing, or 1-Loser 8 with an outside A or K).
                \item We are similarly aggressive with jump overcalls and preemptive raises.
            \end{itemize}
            \item \label{itm:wk2resp2}Weak 2 responses:
            \begin{description}[nosep]
                \item [RONF] Raises are our only Non-Forcing bids, and we are aggressive and Law-abiding with them.
                \item [2\,NT] Ogust.
                      3\,m/M shows bad/good hand, low/high of pair shows bad/good suit.
                      ``Good Suit'' is not specifically defined, it is relative to the worst suit we would open at this seat/vul.
                \item \label{itm:2xdoubled}[After X:]
                \begin{description}
                    \item [All Bids Are Raises.]
                    If not partner's suit, it is asking for a lead of the bid suit should opener be on lead.
                    \item [XX] says ``my suit is better than your suit'' and is a puppet bid.
                    Opener bids the next suit up, and responder places the contract.
                \end{description}
            \end{description}
        \end{enumerate}
        \item \label{itm:carding}Carding:
        \begin{itemize}[nosep]
            \item Leads: standard 4\textsuperscript{th} best, but A for Attitude, K for Kount throughout.
            \item Signals: We play Upside Down Attitude, Count \textbf{and Suit Preference}.
            Suit preference is the default secondary signal, and we play Trump Suit Preference.
            \item Against NT, we play Reverse Smith Echo (if attitude is not known to opening leader's suit, signal to trick 2 is (upside down) attitude in leader's suit, not a signal about declarer's suit).
        \end{itemize}
    \end{enumerate}
\end{scriptsize}
\end{document}
